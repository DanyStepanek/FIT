\documentclass[11pt, titlepage, a4paper, twocolumn]{article}
\usepackage[left=1.5cm,text={18cm, 25cm},top=2.5cm]{geometry}
\usepackage[czech]{babel}
\usepackage[utf8]{inputenc}
\usepackage[T1]{fontenc}
\usepackage{amsthm,amssymb,amsmath,enumerate}
\usepackage{times}



\begin{document}

	\begin{titlepage}
		\begin{center}
			\Huge
			\textsc{Fakulta informačních technologií Vysoké učení technické v Brně}\\[0.4em]
	\vspace{\stretch{0.382}}
	\LARGE{Typografie a publikování\,--\,2. projekt\\[0.3em]}
	Sazba dokumentů a matematických výrazů
			\vspace{\stretch{0.618}}
	
		\end{center}

{\LARGE 2020 \hfill Daniel Štěpánek (xstepa61)}
	\end{titlepage}




\section*{Úvod}
V~této úloze si vyzkoušíme sazbu titulní strany, matematických vzorců, prostředí a~dalších textových struktur obvyklých pro technicky zaměřené texty (například rovnice (\ref{2}) nebo \ref{def_2} na straně \pageref{def_2}). Pro vytvoření těchto odkazů používáme příkazy \verb|\label|,\verb|\refa\pageref|. Na titulní straně je využito sázení nadpisu podle optického středu s~ využitím zlatého řezu. Tento postup byl probírán na přednášce. Dále je použito odřádkování sezadanou relativní velikostí 0.4em a 0.3em.

\section{Matematický text} 
Nejprve se podíváme na sázení matematických symbolů a~výrazů v~ plynulém textu včetně sazby definic a~ vět s využitím balíku amsthm. Rovněž použijeme poznámku podčarou s~použitím příkazu \verb|\footnote|. Někdy je vhodné použít konstrukci \verb|${}$|nebo \verb|\mbox{}|která říká, že (matematický) text nemá být zalomen. V~následující definici je nastavena mezera mezi jednotlivými položkami\verb|\item| na 0.05em.\\[0.4em]
\textbf{Definice 1.}\label{def_1} Turingův stroj (TS) \emph{je definován jako šestice tvaru ${M = (Q, \Sigma,\Gamma, \delta, q_0, q_F)}$, kde:}
\begin{itemize}\itemsep=0.05em
\item \emph{Q je konečná množina} vnitřních (řídicích) stavů,
\item ${\Sigma}$ \emph{je konečná množina symbolů} nazývaná vstupní abeceda, ${\Delta \,{\not \in}\,\Sigma,}$ 
\item ${\Gamma}$ \emph{je konečná množina symbolů}, ${\Sigma \subset \Gamma, \Delta \in \Gamma}$, \emph{nazývaná} pásková abeceda,
\item ${\delta: (Q\backslash \{q_F\})\hspace{-1mm}\times\hspace{-1mm} \Gamma\rightarrow Q\hspace{-1mm}\times\hspace{-1mm}(\Gamma \cup \{L,R\}), kde L,R\,{\not \in}\,\Gamma,}$ \emph{je parciální} přechodová funkce, a
\item ${q_0 \in Q}$ je počáteční stav a~${q_f \in Q}$ je koncový stav.
\end{itemize}

Symbol ${\Delta}$ značí tzv.\, \emph{blank}(prázdný symbol), který se vyskytuje na místech pásky, která nebyla ještě použita.\par \emph{Konfigurace} pásky se skládá z nekonečného řetězce, který reprezentuje obsah pásky a~pozice hlavy na tomto řetězci. Jedná se o~prvek množiny ${\{\gamma \Delta^\omega | \gamma \in \Gamma \star\} \times \mathbb{N}}$\footnote{Pro libovolnou abecedu $\Sigma$ je $\Sigma ^\omega$ množina všech nekonečných řetězců nad $\Sigma$ , tj. nekonečných posloupností symbolů ze $\Sigma$.}. \emph{Konfiguraci} pásky obvykle zapisujeme jako ${\Delta xyzzx \Delta}$\dots (podtržení značí pozici hlavy). \emph{Konfigurace stroje} je pak dána stavem řízení a~konfigurací pásky. Formálně se jedná o~prvek množiny ${Q\times\{\gamma \Delta^ \omega | \gamma \in \Gamma \star \} \times \mathbb{N}}$.

\subsection{Podsekce obsahující větu a odkaz}
\textbf{Definice 2.} \label{def_2} 
Řetězec ${\omega}$ nad abecedou ${\Sigma}$ je přijat TS \emph{M jestliže M při aktivaci z počáteční konfigurace pásky} ${\underline{\Delta } \omega \Delta }$\dots \emph{ a počátečního stavu} ${q_0}$ \emph{zastaví přechodem do koncového stavu} ${q_F}$,  tj. (${q_0, \Delta \omega \Delta^ \omega , 0 }$ ) ${\vdash^*_M}$ (${q_F, \gamma , n}$) \emph{pro nějaké} ${\gamma \in \Gamma ^\star \  a \  n \in \mathbb{N}}$. \par
\emph{Množinu} ${L(M) = \{\omega \  | \  \omega}$ \emph{je přijat TS M} ${\subseteq \Sigma^\star }$ \emph{nazýváme} jazyk přijímaný TS $M$.\\ \par
Nyní si vyzkoušíme sazbu vět a~důkazů opět s~použitím balíku \verb|amsthm|.\\[0.4em]
\textbf{Věta 1.} \label{veta_1} \emph{Třída jazyků, které jsou přijímány TS, odpovídá} rekurzivně vyčíslitelným jazykům.\\[0.4em]
\emph{Důkaz}. V důkaze vyjdeme z Definice 1 a~2.

\section{Rovnice}
Složitější matematické formulace sázíme mimo plynulý text. Lze umístit několik výrazů na jeden řádek, ale pak je třeba tyto vhodně oddělit, například příkazem \verb|\quad|.\\ \par
\quad ${\sqrt[i]{x_i^3}}$ kde $x_i$ je i-té sudé číslo ${y_i^{2-y_i} \ne y_i^{y_i^{y_i}}}$
\quad \\[0.4em] \par
V rovnici \ref{rov_1} jsou využity tři typy závorek s různou explicitně definovanou velikostí.
\begin{eqnarray} 
\label{rov_1}
x&=&\bigg\{\Big( \big[ a + b\big] \Big)^d \oplus 1 \bigg\}\\
\label{rov_2}
y&=& \lim_{x \to \infty} \frac{\sin^2x + \cos^2x}{\frac{1}{\log_{10} x}}
\end{eqnarray}
V~této větě vidíme, jak vypadá implicitní vysázení limity $\lim{n \to \infty} f(n)$ a~v~normálním odstavci textu. Podobně je to i~s~dalšími symboly jako $\sum_{i=1}^n 2^i$ či $\bigcap_{A \in B} A $. V případě vzorců $\lim\limits_{n \to \infty} f(n) \sum\limits_{i=1}^n 2^i$ jsme si vynutili méně úspornou sazbu příkazem \verb|\limits|.
\section{Matice}
Pro sázení matic se velmi často používá prostředí array a~závorky (\verb|\left|, \verb|\right|).\\
$$ \left(\begin{array}{ccc} 
a+b&\widehat{\xi+\omega}&\hat{\pi} \\
\vec{a}&\overleftrightarrow{AC}&\beta\\
\end{array} \right) = 1 \Longleftrightarrow \mathbb{Q} = \mathbb{R}$$
\\
Prostředí \verb|array| lze úspěšně využít i~jinde.\\
$$ \binom{n}{k} = 
\left\{\begin{array}{cc}
\ 0&\text{pro } k < 0 \text{ nebo } k > n\\
\ \frac{n!}{k!(n-k)!}&\text{pro } 0 \leq k \leq n.
\end{array} \right. $$



\end{document}
