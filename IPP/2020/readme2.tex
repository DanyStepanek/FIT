\documentclass[a4paper, 10pt]{article}

\usepackage[left=2cm, text={17cm, 23cm}, top=2cm]{geometry}
\usepackage[czech]{babel}
\usepackage[utf8x]{inputenc}
\usepackage[T1]{fontenc}
\usepackage{alltt}
\usepackage{times}

\begin{document}

\hspace{-5mm}Implementační dokumentace k 2. úloze do IPP 2019/2020\\
Jméno a příjmení: Daniel Štěpánek\\
Login: xstepa61\\
\\
\textbf{\Large{Postup implementace}}

\section{Interpret (interpret.py)}

\subsection{Struktura programu}
\qquad Program je objektově navržený. Není využit žádný návrhový vzor. Překlad a interpretace je rozdělena do tří kroků. V prvním kroku probíhá kontrola vstupního XML a načtení instrukcí. Ve druhém kroku probíhá syntaktická analýza pomocí regulárních výrazů. Ve třetím kroku probíhá sémantická analýza a samotná interpretace.

\subsection{Spuštění skriptu}
\qquad Při spuštění skriptu je provedena kontrola vstupních parametrů. Pokud je vstupni parametr  --help, je vypsána nápověda ke skriptu a~následné ukončení vykonávání. Jinak, v~případě správné kombinace parametrů, s využitím knihovny 'xml.etree.ElementTree' je provedeno parsování vstupního XML kódu. Jednotlivé instrukce jsou uloženy podle pořadového čísla \emph{order} do slovníku. 
	
\subsection{Zpracování instrukcí}
\qquad Nejprve je porovnán operační kód instrukce s klíči ve slovníku instrukcí ve třídě InstructionDict, tedy zda operační kód existuje a je správně zapsán. Pokud je operační kód správný, pomocí informací přiřazených operačnímu kódu ve slovníku klíčových slov, se určí počet argumentů a typy jednotlivých argumentů (var, symb, label, type). Jednotlivé argumenty jsou zkontrolovány regulárnímy výrazy. 

\subsection{Interpretace kódu}
\qquad Interpret využívá dvou průchodů programem. Prvním průchodem jsou uložena všechna návěští do slovníku návěští. Druhý průchod je samotné provádění programu. Pro každou instrukci je vytvořena funkce ve třídě \emph{Processor}. Tyto funkce jsou postupně volány v hlavní smyčce podle aktuální hodnoty order. Provádění programu je implicitně sekvenční, případně lze ovlivnit skokovými instrukcemi. 

\subsection{Třídy Frame, Stack}
\qquad Pro práci s~rámci je vytvořena třída \emph{Frame}, jejíž implementace je řešena pomocí slovníku, s~vlastnímy metodami. \\
Zásobník volání (tzv. call stack) i~datový zásobník jsou instance třídy \emph{Stack}. Jenž je implementovaná dynamickým polem s~vlastnímy metodami. 


\newpage

\section{Testovací skript (test.php)}

\subsection{Spuštění skriptu}
\qquad Při spuštění skriptu je provedena kontrola vstupních parametrů. Pokud je vstupni parametr  --help, je vypsána nápověda ke skriptu a~následné ukončení vykonávání. Jinak, v~případě správné kombinace parametrů, program provádí testování.

\subsection{Načítání testovacích souborů}
\qquad Pokud je nastaven volitelný parametr --directory, je prohledávána zadaná složka a~načítají se všechny soubory s~příponou '.src'. Pomocí parametru --recursive, lze zajistit hledání souborů v~podadresářích. Implicitně probíhá hledání souborů v~aktuálním adresáři. 

\subsection{Testování skriptu parse.php (-\,-parse-only)}
\qquad Explicitně lze nastavit cestu k~parse.php pomocí parametru -\,-parse-script, jinak je použit (pokud je vytvořen) skript parse.php v~aktuálním adresáři. \newline
Pro vyhodnocení testování jsou použity nástroje \emph{diff} (s parametry -q -b) pro kontrolu návratových kódů a nástroj \emph{jexamxml} pro kontrolu vygenerovaného XML s referenčním XML. Explicitně lze nastavit cestu k nástroji jexamxml pomocí parametru -\,-~jexamxml.


\subsection{Testování skriptu interpret.py (-\,-int-only)}
\qquad Explicitně lze nastavit cestu k~interpret.py pomocí parametru -\,-int-script, jinak je použit (pokud je vytvořen) skript interpet.py v~aktuálním adresáři. \newline
Pro vyhodnocení testování je použit nástroj \emph{diff} (s parametry -q -b) pro kontrolu návratových kódů i výstupu ze skriptu s referenčím výstupem.

\subsection{Test celého překladu}
\qquad V první části je provedeno parsování vstupního souboru s příponou '.src' skriptem parse.php, jenž vygeneruje XML kód přesměrovaný do souboru s příponou '.par'. Tento soubor je nyní vstupem pro skript interpret.py, jehož výstup je přeměrován do souboru s příponou '.int' a následně je porovnán výstup i návratový kód v souboru s připonou '.intrc' s referenčními soubory pomocí nástroje \emph{diff}. 

\subsection{Generování HTML}
\qquad Výstupem programu je generovaný HTML kód na standardní výstup. Pro každý test je dle úspěchu vygenerován HTML kód ve tvaru: Test passed/failed cesta k souboru. Je použito jednoduché stylovaní pomocí CSS.
 
\end{document}

