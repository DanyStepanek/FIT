\chapter{Úvod}
S~rostoucím množstvím chytrých zařízení v~našich životech roste i~potřeba stále zlepšovat interakci mezi člověkem a~počítačem. Mezi nejpoužívanější techniky patří dotazníky, pozorování uživatele při práci s~uživatelským rozhraním apod. V posledních letech se výzkum zaměřuje na využití emocí a~jejich, co nejsnadnější detekci. Díky nositelné elektronice (chytré hodinky, chytré náramky, hrudní pásy pro měření fyziologických dat apod.) a~její velké popularitě, je zajímavé ověřit, zda se dají emoce uživatele detekovat pomoci těchto zařízení. Pro tuto práci byl zvolen náramek E4 společnosti Empatica.

Cílem je poskytnout uživatelům nástroj, který umožní sledovat jejich emoce, a~díky tomu zlepšit uživatelskou zkušenost v~reálném čase. Dále může práce najít využití při uživatelských testech, kdy by moderátor dostával informace o~vnitřním stavu subjektu a~mohl by přizpůsobit průběh experimentu. Cílem práce je navrhnout experiment, získat a~zpracovat datovou sadu. Tato sada, včetně skriptů, je dostupná na platformě GitHub\footnote{\url{https://github.com/DanyStepanek/FIT/tree/master/IBT}}.

V~kapitole~\ref{reserse} jsou blíže specifikovány podstatné pojmy z~psychologie, uživatelská zkušenost, fyziologická data, jejich spojitost s~nervovou soustavou člověka a~možnosti měření. V~kapitole~\ref{navrh} je popsán návrh experimentu, použité podněty pro vyvolání emocí a~pracovní prostředí. Kapitola~\ref{experiment} pojednává o~realizaci a~průběhu experimentu. V~kapitole~\ref{vysledky} je popsána získaná datová sada a~její anotace, včetně zhodnocení kvality dat. Výsledky práce i~jejich zhodnocení jsou popsány v~kapitole~\ref{zaver}.  