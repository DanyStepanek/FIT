\pagebreak
    \chapter{Závěr}
    \label{zaver}
    Cílem práce bylo navrhnout a~provést experiment za účelem získaní fyziologických dat pro vybrané emoce. Při řešení byly nejprve nastudovány potřebné teoretické základy a~již existující řešení. Studovaná literatura se týkala emotivity člověka a~jejího vlivu na autonomní nervový systém. Dále byly prostudovány fyziologické funkce i~možnosti jejich měření.
    
    Během návrhu se postupovalo z~nastudovaných poznatků, především existujících řešení, podle kterých byly vybrány typy podnětů, jenž by bylo možné využít. Typicky jsou experimenty založeny na jednom typu podnětů. V~této práci byla zvolena kombinace tří typů (obrázky, videa, zvuky). Výběr podnětů se řídil kategorizačním skóre. Volba zařízení pro měření fyziologických dat závisela především na vybraných metrikách a~způsobu měření. Při návrhu byly stanoveny dvě podstatné fyziologické funkce (srdeční tep, galvanická odezva kůže). Způsob měření musel splňovat dva požadavky: minimálně rušivé měření a~být neinvazivní. Podle kritérií nakonec bylo vybráno zařízení Empatica E4. Zpracování dat probíhalo v~jazyce Python\,3.7 s~využitím knihoven Pandas a~NeuroKit2.
    
    Experiment probíhal v~jednotlivých sezeních a~zúčastnilo se jej 16 dobrovolníků (9~mužů, 7~žen). Před ostrým testováním proběhlo několik pilotních testů k~ověření funkčnosti jednotlivých skriptů i~plynulého průběhu měření. Testy odhalily chyby, jenž byly následně opraveny. 
    
    Výsledkem práce je datová sada fyziologických dat, kategorizována podle pohlaví účastníka experimentu a~pocítěné emoce. I~přes poměrnou chybovost získaných dat, kde chybovost měření srdečního tepu dosahovala průměrně~45\,\%, byl potvrzen rozdílný vliv jednotlivých emocí na zvolené fyziologické funkce člověka. Značnou chybovost zapříčinil pravděpodobně špatný kontakt zařízení s~povrchem těla testovaného, či nechtěný pohyb. K~tomu docházelo i~přes to, že byli všichni účastníci předem poučeni, aby seděli nehybně a~náramek si pevně nasadili. Dalším aspektem je velká citlivost PPG senzoru na vnější vlivy.
    
    Ačkoliv existují podobné studie, snahou bylo získat data pomocí zařízení, co nejvíce blízkému běžné nositelné elektronice. Oproti jiným studiím byl zvolen jiný přístup k~volbě podnětů. Typicky je využit pouze jeden typ podnětů, kdežto v~této práci byla zvolena kombinace tří typů.