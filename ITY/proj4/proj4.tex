\documentclass[a4paper, 11pt]{article}

\usepackage[czech]{babel}
\usepackage[utf8]{inputenc}
\usepackage[left=2cm, top=3cm, text={17cm, 24cm}]{geometry}
\usepackage{times}
\usepackage[unicode]{hyperref}
\usepackage[numbers]{natbib}
\usepackage{url}
\DeclareUrlCommand\url{\def\UrlLeft{<}\def\UrlRight{>} \urlstyle{tt}}
\hypersetup{colorlinks = true, hypertexnames = false}

\begin{document}

	\begin{titlepage}
		\begin{center}
			\Huge
			\textsc{Vysoké učení technické v~Brně} \\
			\huge
			\textsc{Fakulta informačních technologií} \\
			\vspace{\stretch{0.382}}
			\LARGE
			Typografie a~publikování\,--\,4.~projekt \\
			\Huge
			Bibliografické citace
			\vspace{\stretch{0.618}}
		\end{center}

{\LARGE \today \hfill Daniel Štěpánek (xstepa61)}
	\end{titlepage}

\section{Co to je \LaTeX}
\LaTeX \ je balík maker programu \TeX \ jež v~roce 1994 vytvořil Leslie Lamport. \LaTeX \ je založen na sázecím systému \TeX, který v~roce 1983 vytvořil Donald E. Knuth za účelem zlepšení úrovně typografie a~také eliminace některých chyb při sázení \cite{Kopkac2004}\cite{Simecek2013}. Nespornou výhodou \LaTeX u je také to, že ho lze používat zadarmo.

\section{Co \LaTeX \ dokáže}
Hlavní výhodou používání \LaTeX u pro sázení textu je výsledná kvalita sazby dokumentů, nezávislost na operačním systému, flexibilita použití a~podpora pro různé specializované oblasti jako např. sázení matematických rovnic a~odborného textu.


\subsection{Struktura dokumentu}
Dokument v~{\LaTeX}u je rozdělen na dvě části. První část (preambule) obsahuje globální nastavení dokumentu.
Druhá část už obsahuje vlastní text viz \cite{Rybicka2003}.


\section{Kde hledat další informace}
Informací o~\LaTeX u, \TeX u a~typografii samotné je na internetu nepřeberné množství. Lze také nalézt serialové publikace\cite{MackiewiczJ2003Wtws} a~různé články\cite{DemaineErikD.2015FwfA}\cite{BangWong2011PovT}, zabývající se touto tématikou. Po zvládnutí základů je samozřejmě možné pokračovat dál a~objevovat pokročilé možnosti tohoto nástroje \cite{Svamberg2001}\cite{Martinek2010}.

\subsection{Odborné práce}
Na téma typografie a~samotný \LaTeX \ také vzniklo mnoho odborných prací. Několik z~nich dokonce na naší fakultě viz \cite{Sokol2012} \cite{Simek2009}. \cite{test}

\newpage
\bibliographystyle{czechiso}
\renewcommand{\refname}{Literatura}
\bibliography{proj4}

\end{document}
